\documentclass{beamer}
\usetheme{default}
%\usetheme{pittsburgh}
\usecolortheme{albatross}

%###############################################################################
%#
%# Saját színek:
%#
\definecolor{todobgszin}{rgb}{0.64000,0.78000,0.22000}
\definecolor{todofrszin}{rgb}{0.00000,0.50000,0.00000}
\definecolor{background}{rgb}{0.00000,0.29412,0.49804}
%#
%###############################################################################

\setbeamercolor{normal text}{fg=white}
\setbeamertemplate{navigation symbols}{} % Navigációs ikonok off

\usepackage[T1]{fontenc}
\usepackage[utf8]{inputenc}
\usepackage[english,magyar]{babel}

\usepackage{hyperref}
\usepackage{color}
\usepackage{graphics}

%# Hogy lehessen blokkokat megjegyzéssé tenni:
\usepackage{verbatim}

\usebackgroundtemplate{
\includegraphics[width=\paperwidth, height=\paperheight]{figures/background.jpg}
}

% Néhány konstans deklarációja:
\newcommand{\vikszerzo}{Nádudvari György}
\newcommand{\vikszerzomail}{ulqp9p@gmail.com}
\newcommand{\vikkonzulens}{Gönczy László}
\newcommand{\vikkonzulensmail}{gonczy@mit.bme.hu}
\newcommand{\vikcim}{Virtualizált környezetek teljesítmény elemzése vizuális adatelemzés módszerekkel}

%###############################################################################
%#
%# Footer definíció a szokásos címek beírásához.
%#
%# Csúnya hacket tartalmaz!!!
%# A "{footline}{#1}" utáni \textcolorral egy háttérszínnel megegyező "y"
%# karaktert írunk ki, hogy ne ugráljanak a dobozok a diákon.
%# Ha ez elmarad, akkor az alapvonal alá nyúló karaktereket (pl. g, y stb.)
%# tartalmazó stringek esetén eltérő magasságban lesz a footer, mint azoknál
%# amelyek csak az alapvonalra illeszkedő karaktereket tartalmaznak.
%#
\newcommand{\setfootline}[1]{\setbeamertemplate{footline}{\setbeamercolor{footline}{fg=white}\begin{beamercolorbox}[sep=1cm,wd=\textwidth,ht=1cm,left]{footline}{#1}\textcolor{background}{y}\end{beamercolorbox}}}
%#
%###############################################################################

%###############################################################################
%#
%# Saját eszközök:
%#
\newcommand{\todo}[1]{\fcolorbox{todofrszin}{todobgszin}{\emph{TODO: #1}}}
\newcommand{\angolul}[1]{\foreignlanguage{english}{#1}}
%#
%###############################################################################

\hypersetup{
    bookmarks=true,            % show bookmarks bar?
    unicode=true ,             % non-Latin characters in Acrobat’s bookmarks
    pdftitle={\vikcim},        % title
    pdfauthor={\vikszerzo},    % author
    pdfnewwindow=true,         % links in new window
    colorlinks=true,           % false: boxed links; true: colored links
    linkcolor=black,           % color of internal links
    citecolor=black,           % color of links to bibliography
    filecolor=black,           % color of file links
    urlcolor=black             % color of external links
}

%###############################################################################
%#
%#                     CÍM:
%#
\title{\vikcim}
\author{\vikszerzo \\ {\footnotesize <~\vikszerzomail~>} \\ [0.5cm] {\small Konzulens: \vikkonzulens} \\ {\footnotesize <~\vikkonzulensmail~>}}
\date{2012. december 07.}
%#
%###############################################################################

%###############################################################################
%#
%#                    DOKUMENTUMTÖRZS
%#
\begin{document}

%# Hogy ne kelljen a section-nek és a frametitle-nek is megadni ugyanazt a 
%# címet:
\newcommand{\slidecim}{}

%###############################################################################
%#
%#                    1. DIA:
%#
\section{Cím lap - \vikcim}
\begin{frame}[plain]
\titlepage
\end{frame}
%#
%###############################################################################

%###############################################################################
%#
%#                    MIRŐL IS LESZ SZÓ?
%#
\renewcommand{\slidecim}{Miről is lesz szó?}
\section{\slidecim}
\setfootline{\todo{Keep floyding}}
\begin{frame}[t]
\frametitle{\slidecim}
\begin{itemize}
    \item Feladatom a félévben
    \item Az adatelemzés általános áttekintése
    \item A vizualizáció előnyei, hátrányai
    \item VMware virtualizációs technológia
    \item Virtualizációs metrika adathalmazok vizualizációja
    \item Összefoglalás
\end{itemize}
\end{frame}
%#
%###############################################################################

%###############################################################################
%#
%#                    MI VOLT A FELADATOM A FÉLÉV SORÁN?
%#
\renewcommand{\slidecim}{Mi volt a feladatom a félév során?}
\section{\slidecim}
\setfootline{\todo{Keep floyding}}
\begin{frame}[t]
\frametitle{\slidecim}
\begin{itemize}
\item Az alap tudás elsajátítása:
    \begin{itemize}
        \item R nyelv
        \item adattisztítás, adatelemzés
        \item grafikus vizualizáció
        \item VMware metrikák
    \end{itemize}
\item Az elsajátított tudás alkalmazása infrastruktúrákból származó naplók (esemény, metrika) feldolgozására, elemzésére
\item Az aktuális félévben:
    \begin{itemize}
        \item Saját tanszéki virtualizációs infrastruktúrából származó metrikák vizuális elemzése
    \end{itemize}
\end{itemize}
\end{frame}
%#
%###############################################################################

%###############################################################################
%#
%#                    MI AZ, HOGY ADATELEMZÉS?
%#
\renewcommand{\slidecim}{Mi az, hogy adatelemzés?}
\section{\slidecim}
\setfootline{\todo{Keep floyding}}
\begin{frame}[t]
\frametitle{\slidecim}
\begin{itemize}
    \item Adott:
        \begin{itemize}
            \item nagy méretű adathalmaz,
            \item háttértudás, információ a vizsgált területről,
        \end{itemize}
    \item Adatgyűjtés, majd adattisztítás után
    \item Elemzés:
        \begin{itemize}
            \item új információk szerzése,
            \item új következtetések levonása,
            \item vagyis feltáró adatelemzés.
        \end{itemize}
    \item Használható technikák:
        \begin{itemize}
            \item klaszterezés,
            \item simító, regressziós görbék,
            \item stb.
        \end{itemize}
\end{itemize}
\end{frame}
%#
%###############################################################################

%###############################################################################
%#
%#                    AZ ADATVIZUALIZÁLÁS ELŐNYEI, KIHÍVÁSAI
%#
\renewcommand{\slidecim}{Az adatvizualizáció előnyei, kihívásai}
\section{\slidecim}
\setfootline{\todo{Keep floyding}}
\begin{frame}[t]
\frametitle{\slidecim}
\begin{itemize}
    \item Előnyök:
        \begin{itemize}
            \item Nagy adatmennyiség könnyű áttekinthetősége
            \item Hibák könnyű felfedezése
            \item Léteznek eszközök, amelyek képesek diagram típusok között kapcsolatokat létrehozni
            \begin{itemize}
                \item ilyen pl. a Mondrian vizuális adatelemző
            \end{itemize}
        \end{itemize}
    \item Kihívások:
        \begin{itemize}
            \item figyelni kell az eltérő skálákra
            \item előfordulhat, hogy ott találunk új információt, ahol nincs is
        \end{itemize}
\end{itemize}
\end{frame}
%#
%###############################################################################

%###############################################################################
%#
%#                    EGY PÉLDA - ...
%#
\renewcommand{\slidecim}{Egy példa - Zipf eloszlás megjelenése a kutatócsoport honlapjának látogatottságában}
\section{\slidecim}
\setfootline{\todo{Keep floyding}}
\begin{frame}[t]
\frametitle{\slidecim}
\begin{figure}[!ht]
\centering
\includegraphics[height=68mm, keepaspectratio]{figures/zipf_url_views_cat_plot.png}
\end{figure}
\end{frame}
%#
%###############################################################################

%###############################################################################
%#
%#                    MÉG EGY PÉLDA - ...
%#
\renewcommand{\slidecim}{Még egy példa - Zipf eloszlás megjelenése a Rendszermodellezés tárgy aloldalainak látogatottságában}
\section{\slidecim}
\setfootline{\todo{Keep floyding}}
\begin{frame}[t]
\frametitle{\slidecim}
\begin{figure}[!ht]
\centering
\includegraphics[width=110mm, keepaspectratio]{figures/zipf_remo_url_views_cat_plot.png}
\end{figure}
\end{frame}
%#
%###############################################################################

%###############################################################################
%#
%#                    A VMWARE VIRTUALIZÁCIÓS TECHNOLÓGIA METRIKÁI
%#
\renewcommand{\slidecim}{A VMware virtualizációs technológia metrikái}
\section{\slidecim}
\setfootline{\todo{Keep floyding}}
\begin{frame}[t]
\frametitle{\slidecim}
\begin{itemize}
\item VMware metrikák:
    \item 20 másodperces mérési ablakok
    \item Típusok:
        \begin{itemize}
            \item átlagok
            \item összegek
            \item minimumok, maximumok
            \item százalékok
        \end{itemize}
    \item Eszközök:
        \begin{itemize}
            \item CPU metrikák (cpu.*),
            \item memória metrikák (mem.*)
            \item lemez metrikák (disk.*, virtualdisk.*)
            \item hálózati metrikák (net.*)
            \item egyéb metrikák (pl. sys.uptime.latest, storagepath.*, storageadapter.*)
        \end{itemize}
\end{itemize}
\end{frame}
%#
%###############################################################################

%###############################################################################
%#
%#                    VIRTUALIZÁCIÓS INFRASTRUKTÚRA METRIKÁK VIZUALIZÁCIÓJA
%#
\renewcommand{\slidecim}{Virtualizációs infrastruktúra metrikák vizualizációja}
\section{\slidecim}
\setfootline{\todo{Keep floyding}}
\begin{frame}[t]
\frametitle{\slidecim}
\todo{•}
\end{frame}
%#
%###############################################################################

%###############################################################################
%#
%#                    LEMEZ KÉSLELTETÉSEK ÖSSZEFÜGGÉSEI
%#
\renewcommand{\slidecim}{Lemezkésleltetések összefüggései}
\section{\slidecim}
\setfootline{\todo{Keep floyding}}
\begin{frame}[t]
\frametitle{\slidecim}
\begin{itemize}
    \item Mért metrikák:
        \begin{itemize}
            \item disk.deviceLatency.average, disk.kernelLatency.average, disk.totallatency.average
        \end{itemize}
    \item Kapcsolatuk:
        \begin{itemize}
            \item $totallatency = deviceLatency + kernelLatency$
        \end{itemize}
\end{itemize}
\todo{Képet újragenerálni kisebb intervallumra!}
\begin{figure}[!ht]
\centering
\includegraphics[width=110mm, keepaspectratio]{figures/disk_metrics_sumlatency.png}
\end{figure}
\end{frame}
%#
%###############################################################################

%###############################################################################
%#
%#                    HIBÁS ADATFELDOLGOZÁS, HIBÁS KÖVETKEZTETÉSEK
%#
\renewcommand{\slidecim}{Hibás adatfeldolgozás $\rightarrow$ hibás következtetések}
\section{\slidecim}
\setfootline{\todo{Keep floyding}}
\begin{frame}[t]
\frametitle{\slidecim}
\begin{itemize}
    \item Mért metrika: sys.uptime.latest
\end{itemize}
\vspace{-20pt}
\begin{figure}[!ht]
\centering
\includegraphics[width=110mm, keepaspectratio]{figures/all_metrics_sys_uptime_latest.png}
\end{figure}
\vspace{-30pt}
\begin{figure}[!ht]
\centering
\includegraphics[width=110mm, keepaspectratio]{figures/all_metrics_net_received_average_vusb.png}
\end{figure}
\end{frame}
%#
%###############################################################################

%###############################################################################
%#
%#                    ÖSSZEFOGLALÁS - ELÉRT EREDMÉNYEK
%#
\renewcommand{\slidecim}{Összefoglalás - Elért eredmények}
\section{\slidecim}
\setfootline{\todo{Keep floyding}}
\begin{frame}[t]
\frametitle{\slidecim}
\begin{itemize}
\item Megismert technológiák, eljárások:
    \begin{itemize}
        \item R statisztikai szkript nyelv alapjai
        \item Adattisztítás, adatelemzés alapjai
        \item VMware virtualizációs infrastruktúra metrikák
    \end{itemize}
\item Eredmények:
    \begin{itemize}
        \item VMware ESX mérésének ellenőrzése, mérési hibák létezésének megmutatása
        \item Gyűjtött metrikák számának lehetséges csökkentése redundáns információk megtalálásával
        \item A Rendszermodellezés tárgy 5., naplóelemzéssel kapcsolatos gyakorlatának kidolgozása
    \end{itemize}
\item Továbblépési lehetőségek:
    \begin{itemize}
        \item Naplóállomány feldolgozásának, tisztításának automatizálása
        \item Virtualizációs infrastruktúrák teljesítmény problémáinak előrejelzése, automatikus felderítése
    \end{itemize}
\end{itemize}
\end{frame}
%#
%###############################################################################

%###############################################################################
%#
%#                    KÖSZÖNETNYÍLVÁNÍTÁS
%#
\renewcommand{\slidecim}{Köszönetnyilvánítás}
\section{\slidecim}
\setfootline{\todo{Keep floyding}}
\begin{frame}[c]
\frametitle{\slidecim}
Köszönet illeti
\begin{center}
{\Large \textit{Micskei Zoltánt}},
\end{center}
aki rendelkezésemre bocsájtotta a tanszéki virtualizációs infrastruktúra metrikáit, és tanácsaival segítette munkámat.
\end{frame}
%#
%###############################################################################

%###############################################################################
%#
%#                    KÉRDÉSEK
%#
\section{Kérdések?}
\setfootline{\angolul{Is There Anybody Out There?}}
\begin{frame}[c]
\frametitle{}
\begin{center}
\huge{\textbf{Kérdések?}}\\
\begin{figure}[!ht]
\centering
\includegraphics[width=20mm, keepaspectratio]{figures/questions.png}
\end{figure}
\end{center}
\end{frame}
%#
%###############################################################################

%###############################################################################
%#
%#                    THE END
%#
\section{Vége}
\setfootline{\angolul{Comfortably Numb}}
\begin{frame}[c]
\frametitle{}
\begin{center}
\huge{\textbf{Köszönöm a figyelmet!}}
\end{center}
\end{frame}
%#
%###############################################################################

%###############################################################################
%#
%#                    SABLON DIA
%#
\begin{comment}

\renewcommand{\slidecim}{•}
\section{\slidecim}
\setfootline{\todo{Keep floyding}}
\begin{frame}[c]
\frametitle{\slidecim}
\todo{•}
\end{frame}

\end{comment}
%#
%###############################################################################

\end{document}
